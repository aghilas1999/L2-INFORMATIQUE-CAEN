\documentclass[12pt]{article}

\usepackage[utf8]{inputenc}%encodage des caractères
\usepackage[T1]{fontenc}%encodage de la police
\usepackage[french]{babel}%langue française
\usepackage{hyperref}
\usepackage{amsmath}
\usepackage{graphicx}

\title{Mon rapport}
\author{Alain Terrieur}
\date{\today}%date du jour

\begin{document}

\maketitle
\thispagestyle{empty}
\setcounter{page}{0}

\newpage
\tableofcontents
\newpage

\section{Ma première section}

\subsection{Une sous-section}
\label{coucou}

\textit{Ce texte est en italique} tandis que \textbf{celui-ci est en gras}.

\subsection{Une autre sous-section}

Liste des items :

\begin{itemize}
	\item item 1
	\item item 2
	\item item 3
\end{itemize}

\section{Nouvelle section}

Ceci est un rapport rédigé en LaTeX\footnote{langage que nous apprenons aujourd'hui.}. La première sous-section, c'est-à-dire la sous-section~\ref{coucou} se situe en page~\pageref{coucou} (numéro de page calculé automatiquement). Ci-dessous se trouve une liste numérotée d'items :

\begin{enumerate}
	\item bla
	\item ble
	\item bli
	\item blo
	\item blu
\end{enumerate}

\section{Les tableaux}

\begin{table}[ht]
 \begin{center}
	\begin{tabular}{|c|l|r|}
		\hline
		texte centré & texte à gauche & texte à droite  \\\hline
		a & b & c \\\hline
		d & e & f \\\hline
	\end{tabular}
 \end{center}
 \caption{Nom du tableau}
 \label{tab:mon_tableau}
\end{table}

Le tableau \ref{tab:mon_tableau} est nommé.

\section{Insérer une image}

La Figure \ref{fig:mon_image} mesure en largeur la moitié de la largeur du texte.

\begin{figure}[ht]
  \begin{center}
    \includegraphics[width=0.5\textwidth]{./images/smiley.png} 
  \end{center}
  \caption{Une photo de smiley}
  \label{fig:mon_image}
\end{figure}

\section{Le mode mathématique}


Le nombre $\pi$ vaut environ $3.14$, ou encore $\frac{22}{7}$ ou $\frac{\frac{44}{2}}{\sqrt{7^2}}$ à un (gros) $\epsilon$ près.\\



Une équation non numérotée :
$$a^2 + b^2 = c^2$$

Une équation numérotée :

\begin{equation}
a^2 + b^2 = c^2
\label{equation}
\end{equation}

On peut citer l'équation précédente comme étant l'équation \ref{equation}.

Avec "align", on peut citer chaque ligne d'une équation. Par exemple, l'équation~\ref{equation1} et l'équation~\ref{equation2}.

\begin{align}
   f(x) & = & x^2 + 8x + 16 \label{equation1}\\  
   & = & (x+4)^2 \label{equation2}
\end{align}


\section{Citer ses sources}

Je cite la première référence \cite{ref1}. Je peux aussi citer les 3 d'un seul coup \cite{ref1,ref2,ref3}.

%On utilise soit thebibliography, soit l'appel au fichier Sample.bib :

%\begin{thebibliography}{9}
%\bibitem{ref1} A. H. Dekker and B. D. Colbert. Network Robustness and Graph Topology. In Proceedings of the 27th Australasian Conference on Computer Science 26, 359-368, 2004.
%\bibitem{ref2} O. Reingold, S. Vadhan, A. Wigderson, Entropy waves, the Zig-Zag graph product, and new constant-degree expanders, Annals of Mathematics 155 (2002) 157-187.
%\bibitem{ref3} J.M. Xu, Topological Structure and Analysis of Interconnection Networks, Kluwer Academic Publishers, Dordrecht Boston London, 2001.
%\end{thebibliography}

\bibliographystyle{unsrt}%ou plain, abbrv, alpha, apalike, ...
\bibliography{Sample}

%\nocite{*}%pour imprimer toutes les éférences
%\bibliography{Sample_with_question_4}

\end{document}
